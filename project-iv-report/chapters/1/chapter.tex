\setcounter{equation}{0}
\chapter{Introduction}

The first understanding of the temperature of the solar corona came in the 1940s \cite{aschwanden}. To the great surprise of the astrophysicists at the time, it was estimated and later confirmed that the temperature of the solar corona was of magnitude $10^6 \text{K}$ in stark contrast to the $6000 \text{K}$ at the surface of the sun. The mechanism in which the corona reach such temperatures in only a few solar radii has been the subject of great debate over the past 50 years. Two theories have emerged to the top: wave heating/nanoflares and magnetic reconnection. Though the former appears to be a suitable description in regions where there are open magnetic field lines \cite{arnab}, it is the latter that better describes the heating in the much hotter closed magnetic loops. Priest et. al \cite{priest} proposed a potential solution they coined \textit{binary reconnection} characterised by the relative motion of pairs of magnetic fragments in the solar magnetic carpet driving reconnection directly, subsequently heating the corona through turbulent relaxation and the propagation of magnetic waves. It is the lowest order nature of these interaction and hence the likelihood of their appearance in nature that makes binary reconnection a strong candidate for a solution to the outstanding problem. \\
The purpose of this report is to test the strength of a key assumption made in the construction of the binary reconnection theory: that the minority polarity has the majority of its flux going to its corresponding majority polarity. Whilst magnetic fragments appear as opposite and equal pairs, over time evolution larger fragments may break down into smaller ones, two features may coalesce and features can disappear through cancellation of opposite polarity flux \cite{meyer}. Therefore, whether this assumption holds at any given moment and whether it holds in time evolution are important questions in determining the strength of the binary reconnection theory.\\
In chapter 2, we introduce the ideas behind magnetic reconnection. We will begin with the magnetohydrodynamics (hereafter referred to as MHD) equations and develop a framework in which reconnection occurs. Furthermore, we discuss the idea of helicity and its role in the energetics of reconnection. We will prove key results, such as the fact the final relaxation state in the turbulent relaxation of field lines is potential. This will give us a platform on which we will eventually able to produce heating numbers in the binary reconnection theory we can compare against empirical data. \\
Chapter 3 will serve us a motivation for the need of the binary reconnection theory by fleshing out the background of the coronal heating problem and the physics in the solar magnetic carpet. The motion of the magnetic fragments is crucial in the development of the binary reconnection theory. \\
In chapter 4 we will develop the theory of binary reconnection as initially introduced by Priest et. al \cite{priest}, adding details to some of their results. Then, in chapter 5 we test their assumptions and the validity of the theory by constructing a Monte Carlo simulation of the solar magnetic carpet and preforming numerical analyses on the evolution of flux distributions. \\
Finally, in chapter 6 we discuss the impact of our results on the overall strength of the binary reconnection theory, keeping in mind the limitations of our own model.