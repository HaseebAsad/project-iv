The coronal heating problem has been at the forefront of magnetohydrodynamics for the past 50 years. The solar corona reach temperatures of $10\times 10^6 \text{K}$ whereas the sun's surface is around $6000 \text{K}$. The heating mechanism for which these corona reach the heightened temperatures is still an outstanding problem. Magnetic reconnection has been proposed as a potential solution to the problem. In this report, we discuss magnetic reconnection and the theory of binary reconnection. We construct a Monte Carlo model to simulate the solar carpet and use numerical methods to learn about the flux distribution between magnetic fragments. Furthermore, we test the validity of the binary reconnection assumption in late time and the implications of the aforementioned assumption.